\section{Packet Transmission via TCP}
\begin{figure}[H]
	\centering
	\begin{subfigure}{.49\textwidth}
		\centering
		\includegraphics[width=1.1\linewidth]{images/task1/client.png}
		\caption{The client interface with JavaFX}
		\label{fig:clientfx}
	\end{subfigure}%
	\begin{subfigure}{.49\textwidth}
		\centering
		\includegraphics[width=1.1\linewidth]{images/task1/server.png}
		\caption{The server interface with JavaFX}
		\label{fig:serverfx}
	\end{subfigure}
	\caption{The Desktop application, structured with tabs.}
	\label{fig:fx}
\end{figure}

\begin{figure}[H]
	\centering
	\begin{subfigure}{.49\textwidth}
		\centering
		\includegraphics[width=0.75\linewidth]{images/task1/clientAndroid.png}
		\caption{The client interface with JavaFX}
		\label{fig:clientAndroid}
	\end{subfigure}%
	\begin{subfigure}{.49\textwidth}
		\centering
		\includegraphics[width=0.75\linewidth]{images/task1/serverAndroid.png}
		\caption{The server interface with JavaFX}
		\label{fig:serverAndroid}
	\end{subfigure}
	\caption{The Android application, structured with two activities}
	\label{fig:android}
\end{figure}

The screenshots provided in figure \ref{fig:fx} provide an overview of the graphical user interface written for the Desktop or JVM application which  is written with the help of JavaFX. 

Figure \ref{fig:android} summarizes the user interface specific to the Android application. 



\subsection{Implementation of a TCP Client}

The program employs an MVC pattern, separating the source code in three parts: model, view and control. The networking part of the code is bundled in the \textit{communication} package, where the rest of the program only interacts with the \textit{Communicator} class. The strategy pattern is applied at this point which allows to seamlessly switch between underlying protocol, namely TCP and UDP. The functionalities of each of the protocols are bundled in a class implementing the \textit{Connector} class. This ensures that each of the protocol specific implementations offers the needed operations. In the case of the client, the interface specifies a contract where the method \textit{sendMessage} has to be realized as can be seen in \ref{lst:sendMessageI}.

\begin{lstlisting}[language=Java, caption={Interface prescribes a sendMessage method},captionpos=b,label=lst:sendMessageI]
/**
 * Sends message to specified ip-address:port.
 *
 * @param ip      receiver address
 * @param port    and receiver port
 * @param message message to be transported
 */
public void sendMessage(String ip, int port, String message);
\end{lstlisting}

The implementation of the \textit{sendMessage} method for the TCP connector can be seen in listing \ref{lst:sendMessageTCP}. In order to offload this functionality into a seperate thread an \textit{ExecutorService} is used where tasks, can be submitted for execution. This ensures the user interface, running on the main thread, does not freeze even if the sending of a message or the establishing of a connection either takes a long time or is unsuccessful.

\begin{lstlisting}[language=Java, caption={TCP sendMessage implementation},captionpos=b,label=lst:sendMessageTCP]
private ExecutorService exs = Executors.newCachedThreadPool();
public void sendMessage(String ip, int port, String message) {
	logger.log(Level.INFO, "Sending message : " + message + " to: " + ip + ":" + port);
	TCPMessageTask messageTask = new TCPMessageTask(ip, port, new Message(message));
	exs.submit(messageTask);
}
\end{lstlisting}

Listing \ref{lst:TCPMessageTask} shows the implementation of the sending of a TCP message. The try-with-resource block ensures that the socket is automatically closed when reaching the end of the statement. Specifying a timeout for the connecting of the \textit{clientSocket} guarantees that after maximally 2000ms this blocking method continues in execution. Since this is executed in its own thread, the termination of the complete program depends on the termination of this method.

\begin{minipage}{\linewidth}
	\begin{lstlisting}[language=Java, caption={TCP sendMessageTask implementation},captionpos=b,label=lst:TCPMessageTask]
@Override
public void run() {
  //try-with-resources block auto closes the socket
  try (Socket clientSocket = new Socket()) {
    InetAddress addr = InetAddress.getByName(ip)
    InetSocketAddress socketAddr = new InetSocketAddress(addr);
    clientSocket.connect(socketAddr, port), 2000);
    //Initialize an object output stream to send the message object
    try (ObjectOutputStream out = new ObjectOutputStream(clientSocket.getOutputStream())) {
      out.writeObject(message);
      out.flush();
    }
			
    } catch (IOException e ) {
      e.printStackTrace();
    }
}
	\end{lstlisting}
\end{minipage}

\subsection{Implementation of a TCP Server}
\begin{lstlisting}[language=Java, caption={Interface prescribes a sendMessage method},captionpos=b,label=lst:sendMessageI]
/**
 * This method starts a server with the set protocol on specified port. The controller will be used to update the interface, especially with received messages.
 *
 * @param port       Port the server listens to
 * @param controller a controller that can be used to for instance update the GUI, display received messages
 */
public void startServer(int port, ControllerInterface controller);
\end{lstlisting}


